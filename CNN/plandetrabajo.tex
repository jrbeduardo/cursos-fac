\documentclass[a4paper,twocolumn,10pt]{article}
\usepackage{hyperref}
\usepackage[utf8]{inputenc}
\usepackage[spanish]{babel}
\usepackage{graphicx}
\usepackage{geometry}
\usepackage{verbatim}
\geometry{a4paper,
width=19cm,
height=28cm,
top=.5cm,
right=1.5cm,
left=1.5cm,
includeall
}
\setlength{\columnsep}{2cm}
%---------------------------------------------------------------------------------

\setlength{\parindent}{0cm}

%---------------------------------------------------------------------------------

\title{
\includegraphics[scale=0.176]{Escudo-facultad-ciencias-unam.png}\hspace{2cm}
\begin{minipage}[b]{8.4cm}
\begin{flushleft}
\normalsize{\scshape Universidad Nacional Autónoma de México\linebreak}  
{  \scshape Facultad de Ciencias\linebreak}     
   {  \scshape Proyecto I\linebreak} {  \scshape Clasificador de imágenes con redes neuronales\linebreak}{\scshape Plan de Trabajo }
\end{flushleft}
\end{minipage}
  \vspace{.5cm} 
 \hrule height1pt\vspace{.1cm}
    \hrule height1pt
\date{}
}

\renewcommand{\baselinestretch}{1.3}
\begin{document}
\maketitle

\begin{flushleft}
\textbf{Datos del curso}
\end{flushleft}


\noindent \textit{Trimestre:} 2026-2

\noindent \textit{Clave:} 2130035

\noindent \textit{Grupo:} 6013

\noindent \textit{Profesor:} José Eduardo Rodríguez Barrios

\noindent \textit{Ayudante:} 

\noindent \textit{e-mail:} \href{mailto:jrbeduardo@gmail.com}{jrbeduardo@gmail.com} 

\noindent \textit{Horario de clases:} Lunes a Viernes  de  17:00-18:00 



\begin{flushleft}
\textbf{Objetivo del Seminario-Taller}
\end{flushleft}

Este seminario-taller tiene como propósito principal que los estudiantes \textbf{desarrollen un proyecto de titulación} en el área de aprendizaje profundo aplicado a la visión por computadora. Durante el curso, los participantes explorarán arquitecturas modernas como \textbf{Redes Neuronales Convolucionales (CNNs)}, \textbf{Vision Transformers (ViTs)} y enfoques multimodales. El objetivo final es que cada estudiante complete un proyecto integral de clasificación de imágenes que pueda ser utilizado como trabajo de titulación.

A través de este curso, los estudiantes:
\begin{itemize}
    \item Comprenderán los fundamentos y el estado del arte en modelos de aprendizaje profundo para visión por computadora.
    \item Desarrollarán habilidades prácticas en el procesamiento de datos, optimización de modelos y despliegue de soluciones.
    \item Documentarán y presentarán su proyecto de manera profesional, siguiendo estándares académicos.
\end{itemize}


\vspace{2\baselineskip}

\begin{flushleft}
\textbf{Temario Sintético}
\end{flushleft}

El curso está dividido en cuatro fases principales:
\begin{enumerate}
    \item \textbf{Fase 1: Fundamentos y Estado del Arte.} Introducción a arquitecturas clásicas y modernas, como CNNs y ViTs. Definición del problema a resolver.
    \item \textbf{Fase 2: Procesamiento de Datos.} Técnicas de \textit{data augmentation}, transferencia de aprendizaje y uso de modelos preentrenados.
    \item \textbf{Fase 3: Optimización y Evaluación.} Ajuste de hiperparámetros, análisis de interpretabilidad y evaluación de modelos.
    \item \textbf{Fase 4: Despliegue y Documentación.} Creación de una API REST, redacción de un reporte técnico con formato de tesis y presentación final del proyecto.
\end{enumerate}


\begin{flushleft}
\textbf{Evaluación}
\end{flushleft}

 
La evaluación del curso se basa en los siguientes componentes:
\begin{itemize}
    \item \textbf{Presentaciones de Artículos Científicos (20\%):} Los estudiantes expondrán y analizarán artículos relevantes al curso.
    \item \textbf{Avances del Proyecto (30\%):} Entregas parciales que reflejen el progreso en cada fase del proyecto.
    \item \textbf{Documentación Técnica (20\%):} Elaboración de un reporte progresivo que cumpla con los estándares de una tesis.
    \item \textbf{Proyecto Final (30\%):} Entrega del proyecto completo, incluyendo código, modelo entrenado, API funcional y presentación profesional.
\end{itemize}

\noindent Para aprobar el curso, se requiere una asistencia mínima del 80\% y una calificación final de al menos 60\%. El proyecto es individual y debe ser documentado en un repositorio de GitHub.

\begin{flushleft}
\textbf{Pre-requisitos}
\end{flushleft}

\noindent \textbf{Matemáticos:} Álgebra lineal, cálculo, probabilidad, regresión logística (deseable).

\noindent \textbf{Técnicos:} Lectura técnica en inglés, Python, Git/GitHub.

\noindent \textbf{Herramientas:} Google Classroom, repositorio GitHub.

\begin{flushleft}
\textbf{Inscripción a Google Classroom}
\end{flushleft}

Para inscribirse al curso en Google Classroom, sigue estos pasos:

\begin{enumerate}
\item Accede a \href{https://classroom.google.com}{classroom.google.com} con tu cuenta de correo electrónico.
\item Haz clic en el botón \textbf{+} en la esquina superior derecha.
\item Selecciona \textbf{Unirse a una clase}.
\item Ingresa el código de la clase: \textbf{ou24niac}
\item Haz clic en \textbf{Unirse}.
\end{enumerate}

\noindent A través de Google Classroom se publicarán anuncios, materiales del curso, asignaciones y entregas de trabajos.

\end{document}
