\documentclass[a4paper,twocolumn,10pt]{article}
\usepackage{hyperref}
\usepackage[utf8]{inputenc}
\usepackage[spanish]{babel}
\usepackage{graphicx}
\usepackage{geometry}
\usepackage{verbatim}
\geometry{a4paper,
width=19cm,
height=28cm,
top=.5cm,
right=1.5cm,
left=1.5cm,
includeall
}
\setlength{\columnsep}{2cm}
%---------------------------------------------------------------------------------

\setlength{\parindent}{0cm}

%---------------------------------------------------------------------------------

\title{
\includegraphics[scale=0.176]{Escudo-facultad-ciencias-unam.png}\hspace{2cm}
\begin{minipage}[b]{8.4cm}
\begin{flushleft}
\normalsize{\scshape Universidad Nacional Autónoma de México\linebreak}  
{  \scshape Facultad de Ciencias\linebreak}     
   {  \scshape Proyecto I\linebreak} {  \scshape Clasificador de imágenes con redes neuronales\linebreak}{\scshape Plan de Trabajo }
\end{flushleft}
\end{minipage}
  \vspace{.5cm} 
 \hrule height1pt\vspace{.1cm}
    \hrule height1pt
\date{}
}

\renewcommand{\baselinestretch}{1.3}
\begin{document}
\maketitle

\begin{flushleft}
\textbf{Datos del curso}
\end{flushleft}


\noindent \textit{Trimestre:} 2026-2

\noindent \textit{Clave:} 2130035

\noindent \textit{Grupo:} 6013

\noindent \textit{Modalidad:} Presencial

\noindent \textit{Profesor:} José Eduardo Rodríguez Barrios

\noindent \textit{Ayudante:} Francisco Pérez Carbajal

\noindent \textit{e-mail:} \href{mailto:jrbeduardo@gmail.com}{jrbeduardo@gmail.com} 

\noindent \textit{Horario de clases:} Lunes a Viernes  de  17:00-18:00 

\begin{comment}
\noindent \textbf{Ayudante:} Flores Gasca Carlos Enrique (con el trabajarán en el salón de clases los días Viernes)

\noindent \textbf{e-mail: }\href{mailto:cefg.mat@gmail.com}{cefg.mat@gmail.com} 

\noindent \textbf{Horario de asesorías del ayudante}: Lunes y Miércoles de 15:30-16:30  y Viernes de 13:30-14:30 en el cubículo de ayudantes, Edif. AT.
\end{comment}


\begin{flushleft}
\textbf{Objetivo del Seminario-Taller}
\end{flushleft}

Este seminario-taller tiene como objetivo que los estudiantes \textbf{desarrollen un proyecto de titulación} en aprendizaje profundo aplicado a visión por computadora, explorando arquitecturas CNNs, Vision Transformers y enfoques multimodales. Los estudiantes desarrollarán un proyecto completo de clasificación de imágenes que pueda servir como trabajo de titulación.

\begin{flushleft}
\textbf{Temario sintético}
\end{flushleft}

\begin{enumerate}
\item[(i)] \textbf{Fase 1 (Sem. 1-4):} Fundamentos y Estado del Arte. Arquitecturas fundacionales y ViTs. Definición del problema.

\item[(ii)]  \textbf{Fase 2 (Sem. 5-8):} Procesamiento de Datos. Data augmentation, transfer learning, modelos fundacionales.

\item[(iii)] \textbf{Fase 3 (Sem. 9-12):} Optimización y Evaluación. Hiperparámetros, fairness, interpretabilidad.

\item[(iv)] \textbf{Fase 4 (Sem. 13-16):} Deployment. API REST, documentación técnica (formato tesis), presentación.

\end{enumerate}


\begin{flushleft}
\textbf{Evaluación}
\end{flushleft}

\begin{itemize}
\item \textbf{Presentaciones de Papers (20\%):} Exposición de artículos científicos asignados.
\item \textbf{Avances del Proyecto (30\%):} Entregas parciales en cada fase.
\item \textbf{Documentación Técnica (20\%):} Reporte progresivo con formato de tesis.
\item \textbf{Proyecto Final (30\%):} Proyecto completo con código, modelo, API/interfaz, reporte y presentación.
\end{itemize}

\noindent Asistencia mínima 80\%, participación activa. Calificación aprobatoria: 60\%. Proyecto individual. Repositorio de GitHub requerido.

\newpage
\begin{flushleft}
\textbf{Pre-requisitos}
\end{flushleft}

\noindent \textbf{Matemáticos:} Álgebra lineal, cálculo, probabilidad, regresión logística (deseable).
\noindent \textbf{Técnicos:} Lectura técnica en inglés, Python, Git/GitHub.
\noindent \textbf{Herramientas:} Google Classroom, repositorio GitHub.

\end{document}
