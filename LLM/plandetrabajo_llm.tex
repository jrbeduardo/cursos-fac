\documentclass[a4paper,twocolumn,10pt]{article}
\usepackage{hyperref}
\usepackage[utf8]{inputenc}
\usepackage[spanish]{babel}
\usepackage{graphicx}
\usepackage{geometry}
\usepackage{verbatim}
\geometry{a4paper,
width=19cm,
height=28cm,
top=.5cm,
right=1.5cm,
left=1.5cm,
includeall
}
\setlength{\columnsep}{2cm}

\setlength{\parindent}{0cm}

\title{
\includegraphics[scale=0.176]{Escudo-facultad-ciencias-unam.png}\hspace{2cm}
\begin{minipage}[b]{8.4cm}
\begin{flushleft}
\normalsize{\scshape Universidad Nacional Autónoma de México\linebreak}
{\scshape Facultad de Ciencias\linebreak}
{\scshape Proyecto I\linebreak}
{\scshape Introducción a los Large Language Models (LLMs)\linebreak}
{\scshape Plan de Trabajo}
\end{flushleft}
\end{minipage}
\vspace{.5cm}
\hrule height1pt\vspace{.1cm}
\hrule height1pt
\date{}
}

\renewcommand{\baselinestretch}{1.3}
\begin{document}
\maketitle

\begin{flushleft}
\textbf{Datos del curso}
\end{flushleft}

\noindent \textit{Semestre:} 2026-2

\noindent \textit{Grupo:} 6012

\noindent \textit{Profesor:} Francisco Pérez Carbajal

\noindent \textit{e-mail:} \href{mailto:franciscop@ciencias.unam.mx}{franciscop@ciencias.unam.mx}

\noindent \textit{Ayudante:} Christian Gustavo Martínez Ramírez

\noindent \textit{e-mail:} \href{mailto:iChristtiann@me.com}{iChristtiann@me.com}

\noindent \textit{Horario de clases:} Lunes a Viernes de 17:00--18:00

\begin{flushleft}
\textbf{Objetivo del Seminario-Taller}
\end{flushleft}

Este seminario-taller tiene como propósito que los estudiantes \textbf{desarrollen un proyecto de titulación} en el área de \textbf{Modelos de Lenguaje Grande (LLMs)}, integrando fundamentos teóricos con aplicaciones prácticas. El curso busca formar una comprensión sólida de los LLMs: su arquitectura (Transformers), sus técnicas de adaptación (fine-tuning eficiente) y su implementación en tareas reales de PLN (por ejemplo: clasificación, resumen, generación o chatbots).

A través de la lectura crítica de artículos científicos, la síntesis de conocimiento especializado y el desarrollo práctico de un proyecto, los estudiantes:
\begin{itemize}
\item Exploran el estado del arte en modelos de lenguaje, desde modelos fundacionales hasta LLMs contemporáneos.
\item Diseñan e implementan un proyecto aplicado, desde la preparación del corpus hasta la evaluación y el despliegue.
\item Documentan y presentan resultados con rigor técnico y estándares académicos.
\end{itemize}

\vspace{2\baselineskip}

\begin{flushleft}
\textbf{Temario Sintético}
\end{flushleft}

El curso está dividido en cuatro fases principales:
\begin{enumerate}
\item \textbf{Fase 1: Fundamentos y Estado del Arte.} Arquitectura Transformer, atención, embeddings, entrenamiento auto-supervisado. Revisión de modelos fundacionales (BERT/GPT/T5/LLaMA/Mistral). Definición del problema del proyecto.
\item \textbf{Fase 2: Datos, Fine-tuning y Experimentación.} Curación del corpus, tokenización y pipelines. Ajuste eficiente (LoRA/QLoRA/PEFT), entrenamiento y validación. Comparación de enfoques y registro de experimentos.
\item \textbf{Fase 3: Evaluación, Interpretabilidad y Fairness.} Métricas (F1/ROUGE/BLEU/perplejidad según el problema), análisis de errores, robustez, sesgos y uso responsable. Interpretabilidad y diagnóstico del modelo.
\item \textbf{Fase 4: Deployment y Documentación.} Integración del modelo en un demo funcional (API/Chatbot/App), optimización (quantization/pruning si aplica), reporte técnico con formato de tesis y presentación final.
\end{enumerate}

\newpage

\begin{flushleft}
\textbf{Evaluación}
\end{flushleft}

La evaluación del curso se basa en los siguientes componentes:
\begin{itemize}
\item \textbf{Presentaciones de Artículos Científicos (20\%):} Exposición y análisis crítico de artículos asignados, conectando conceptos con el proyecto.
\item \textbf{Avances del Proyecto (30\%):} Entregas parciales por fase (definición del problema, baseline, experimentación, evaluación y despliegue).
\item \textbf{Documentación Técnica (20\%):} Reporte progresivo con estructura tipo tesis (metodología, experimentos y análisis).
\item \textbf{Proyecto Final (30\%):} Entrega integral del sistema (código, modelo, demo/API y presentación profesional).
\end{itemize}

\noindent Para aprobar el curso, se requiere una asistencia mínima del 80\% y una calificación final de al menos 60\%. El proyecto es individual y debe estar documentado en un repositorio de GitHub.

\begin{flushleft}
\textbf{Pre-requisitos}
\end{flushleft}

\noindent \textbf{Matemáticos:} Álgebra lineal, probabilidad y estadística. Conocimientos de aprendizaje de máquina / regresión logística (deseable).

\noindent \textbf{Técnicos:} Lectura técnica en inglés, programación en Python, familiaridad con Git/GitHub. Deseable experiencia con PyTorch y/o Hugging Face.

\noindent \textbf{Herramientas:} Google Classroom, repositorio GitHub.

\begin{flushleft}
\textbf{Inscripción a Google Classroom}
\end{flushleft}

Para inscribirse al curso en Google Classroom, sigue estos pasos:
\begin{enumerate}
\item Accede a \href{https://classroom.google.com}{classroom.google.com} con tu cuenta.
\item Haz clic en el botón \textbf{+} en la esquina superior derecha.
\item Selecciona \textbf{Unirse a una clase}.
\item Ingresa el código de la clase: \textbf{3orn6756}
\item Haz clic en \textbf{Unirse}.
\end{enumerate}

\noindent En Google Classroom se publicarán anuncios, materiales, asignaciones y entregas.

\end{document}
